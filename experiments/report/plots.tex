\batchmode
\documentclass[a4paper]{article}
\usepackage[utf8]{inputenc}
\usepackage{graphicx}
\usepackage{xcolor}
\usepackage{pgffor}
\usepackage{float}
\usepackage[font=small,labelfont=bf]{caption}
% For pandas df tables
\usepackage{booktabs}
\usepackage{multirow}
\usepackage[left=2cm,right=2cm,top=2cm,bottom=2cm]{geometry}


\title{EarlyECTS}

% get dataset names
\begingroup\makeatletter\endlinechar=\m@ne\everyeof{\noexpand}
\edef\x{\endgroup\def\noexpand\plots{\@@input|"ls -m output/plot/*/method-plot.pdf" }}\x

\begin{document}
\foreach \plot in \plots {%
    \begin{figure}[H]%
        \makebox[\textwidth][c]{\includegraphics[width=\textwidth]{\plot}}%
        \caption{Example of the performance of the proposed approach.
        Points represent algorithm configurations in the earliness-accuracy space, colors indicate different algorithms.
        Left: 
        	Current standard approach: manual tuning of only the single hyper-parameter controling the earliness-accuracy tradeoff on single algorithms.
        	It is time consuming, only provides limited control over and no insight into the earliness-accuracy tradeoff.
        	The image shows that tuning the dedicated hyper-parameter of a single algorithm does not necessarily translate to finding a wide range of tradeoff points. 
        	The Pareto set that can be constructed from the configurations of a single algorithm typically provides very little choice of tradeoff points.
        Centre: 
        	Combined automated tuning of all hyper-paramerer and algorithm selection (MO-CASH) finds more and better solutions and gives the user insight into the tradeoff by providing a unified Pareto front.
        Right:
        	Due to the configurator, ill performing algorithms are ignored while well performing algorithms are exploited.
        	This allows us to consider additional algorithms that would not make sense without automated configuration or that are designed with optimisationin mind.
        	These additional algorithms may improve on or fill the gaps left by existing algorithms.
        	This figure shows the addition of a naive 1NN ED fixed-time classifier to the portfolio, slightly improving the Pareto front.}%
    \end{figure}%
    \pagebreak
}
\end{document}
